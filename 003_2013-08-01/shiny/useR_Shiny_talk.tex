%\documentclass[compress]{beamer}

\documentclass{beamer}

\usetheme{Dresden} % Beamer Theme
\usecolortheme{default} % Beamer Color Theme

\usepackage{amsmath}
\usepackage{amssymb}
\usepackage{amsfonts}
\usepackage{graphicx} % to include figures
\usepackage{multimedia}
% \usepackage[all]{xy}

\setbeamertemplate{itemize items}[default]
\setbeamertemplate{enumerate items}[default]

\title{Making Your Toolbox Shine with Shiny}
\author{Mikhail Popov}
\institute[UPMC WPIC]{
	Data Scientist\\
	Neuropsychology Research Program\\
	Western Psychiatric Institute of UPMC
}
\date{}

\begin{document}

	\begin{frame}[plain]
		\titlepage
	\end{frame}
	
	\section{Introduction}
	
	\begin{frame}{What is Shiny?}
		Shiny is an R package developed by the RStudio team that lets you develop an interactive web application in R.
		\vskip 3ex
		You do this with two files:
		\begin{itemize}
			\item \texttt{ui.R} (where you define the user interface)
			\item \texttt{server.R} (where you define the server logic)
		\end{itemize}
		\vskip 3ex
		\textbf{Only need to know R.}\\
		No need to know HTML, PHP, or Javascript/jQuery/AJAX.
	\end{frame}
	
	\begin{frame}[fragile]
	\frametitle{Empty ui.R}
\begin{verbatim}
library(shiny)

shinyUI(pageWithSidebar(
  headerPanel("Application Title"),
  sidebarPanel(
    # Input UI elements go here.
  ),
  mainPanel(
    # Output UI elements go here.
  )
))
\end{verbatim}
	\end{frame}
	
	\begin{frame}[fragile]
	\frametitle{Empty server.R}
\begin{verbatim}
library(shiny)

shinyServer(function(input,output){

  # Server logic goes here.

  # You assign reactive elements to 'output' list.
  # You use elements from the 'input' list.
  
})
\end{verbatim}
	\end{frame}

	\begin{frame}{Getting Started}
		\small{
		\textbf{To install the package}\\
		\texttt{install.packages("shiny",repos="http://cran.rstudio.com/")}\\[3ex]
		\textbf{Extensive tutorial by RStudio}\\
		\texttt{http://rstudio.github.io/shiny/tutorial/}\\[3ex]
		\textbf{Built-in demos}\\
		\texttt{shiny::runExample(example="name")}\\[1ex]
		Valid example names are: 01\_hello, 02\_text, 03\_reactivity, 04\_mpg, 05\_sliders, 06\_tabsets, 07\_widgets, 08\_html, 09\_upload, 10\_download, 11\_timer
		}
	\end{frame}

	\section{Examples}
	
	\begin{frame}{Examples}
		Source Code: https://github.com/bearloga/useR-shiny-talk\\[3ex]
		\textbf{Example 1: Random Walk}\\
				Illustrates the fundamental ideas in Shiny via some basic inputs and outputs.
				\vskip 2ex
		\textbf{Example 2: 2D Kernel Density Estimation}\\
				A slightly more advanced Shiny application (app).
				\vskip 2ex
		\textbf{Example 3: Regression (Car Speed vs Stopping Distance)}\\
				This app introduces conditional panels and tabs to visualize curve fitting methods on [transformed] data.

	\end{frame}
	
	\section{Conclusion}

	\begin{frame}{From Here}
		\textbf{Shiny Server} https://github.com/rstudio/shiny-server\\[1.5ex]
		How-to Guides for Deploying on Amazon EC2:
		\begin{itemize}
			\item http://mpopov.com/post/40976561625/shiny-server-amazon-ec2-guide
			\item http://www.r-bloggers.com/deploying-shiny-server-on-amazon-ec2/
		\end{itemize}
		\vskip 3ex
		\textbf{Extensive tutorial by RStudio}\\
		http://rstudio.github.io/shiny/tutorial/
	\end{frame}

	
	\begin{frame}{Thank You}
%		\begin{center}
			\Large
			Examples Source Code\\
			https://github.com/bearloga/useR-shiny-talk\\[3ex]
			Contact\\
			\large
			Email: mikhail@mpopov.com\\
			Web: www.mpopov.com\\
			Twitter: @bearloga
%		\end{center}
	\end{frame}
	
\end{document}